\documentclass[11pt]{article}
\usepackage[margin=1in]{geometry}
\usepackage[utf8]{inputenc}
\usepackage{enumitem}
\usepackage{adjustbox}
\title{\vspace{-1.5in}}
\author{}
\date{}

\usepackage{fontspec}
\setmainfont{Times New Roman}
% use cambria for math
\usepackage{unicode-math}
\setmathfont{[Cambria-Math.ttf]}
% single line spacing
\usepackage{setspace}
\usepackage{float}

\singlespacing

\begin{document}

\maketitle

\thispagestyle{empty}
\centerline{\underline{\textbf{Personal, Background, and Future Goals Statement}}}


%Outline your educational and professional development plans and career goals.  How do you envision that graduate school will prepare you for a career that allows you to contribute to scientific understanding and broadly benefit society?
\hspace{0.5cm}\\
"Would anyone like to volunteer?"

Those words were followed by silence. In response to a letter from the graduate students, the Department of Astronomy at UVA formed a committee to address concerns of race and representation in academia. However, when the time came for us to select student representatives to do the work, none came forward. In that silence, I became an advocate. It has been three months since I started representing the graduate student body on the Diversity, Equity, and Inclusion (DEI) committee at the University of Virginia. Now, I have become my own advocate for change: the very person I was waiting for in that moment of silence. As I pursue my own enrichment in astrophysics, I will continue to use my position to uplift those who face disproportionately large barriers to achieving the same.

\noindent \textbf{\underline{Intellectual Merit}}

The summer after my first year of undergraduate study at the University of Florida, I sought out every professor I knew in the Astronomy Department and asked if I could research with them. My first position was with Yinan Zhao, a graduate student who worked with Dr. Jian Ge on Magnesium-II dust absorbing quasars. I spent two semesters learning the basics of IDL and Python from Yinan's code and supplemental workshops offered by another graduate student, Ben Kimock. This project introduced me to two programming languages, and demonstrated how effectively research skills can be learned from graduate student mentors. As I began to learn Python and IDL, the computational aspects of the project became particularly appealing to me, and I decided to pursue a higher level of skill in programming to explore this newfound interest.

To augment my progress in coding, I took a computational astrophysics course taught by Dr. Desika Narayanan, exploring Python's advanced applications to astronomy. By the end of the course, I didn't just have a better handle on coding; I was empowered by it. I sought out a research opportunity with Dr. Stephen Eikenberry, investigating ultramagnetic variable stars using the EvryScope All-Sky Telescope. In his group, I was the only undergraduate out of a research team of 10. Time-series analysis was performed with PRESTO (Ransom 2001), a pulsar-finding software suite. Using object-oriented programming, I designed a pipeline such that arbitrary time series data could be processed into evenly-spaced bins that are the proper format for PRESTO. \textbf{I used the data pipeline I wrote to refine estimated periodicities of three ultra-magnetic variable stars, confirming that PRESTO can perform scientifically significant analysis on objects other than pulsars.} I was awarded a research scholarship from the University Scholars Program for my work, and was invited to give a talk at a university-wide symposium. This became the foundation of my senior thesis, for which I was awarded the 2019 departmental thesis award. My work on this project helped me understand the benefit of writing modular, generalized scripts that can be quickly adapted to perform several functions as needed. Since then, I have developed my code into a complete software package, called PRESTOport, for which I am the lead developer and maintainer.

In the summer of 2018, I began my REU at the National Radio Astronomy Observatory under the mentorship of Dr. Adam Ginsburg. Our aim was to characterize the spectral energy distributions of protostars in the W51 star-forming region, and my specific directions were to catalog each source to obtain photometric data. I realized this task could be at least partially automated, so I set about writing code that detected and cataloged each source above a certain threshold. False detections due to noise kept cropping up, so I incorporated a rejection algorithm. As I kept making advancements, I was delighted to see progress, but frustrated that each individual script I wrote was limited in its scope. I decided to develop these scripts into a complete, distributable software package called Dendrocat, which performs automated radio source detection, noise rejection, and aperture photometry for radio observations across several frequencies at a time. Using Dendrocat, I determined the mechanisms driving radio light emission among specific protostars in the W51 star-forming region by the shape of their spectral energy distributions. \textbf{I identified at least once source with a high-frequency spectral turnover of 90 GHz, which indicates the presence of an atypically dense HII region (McClellan et al. 2019).} Since this discovery, Dendrocat has become a central part of a continuing effort to catalog similar sources in star-forming regions. I was awarded funding by the NRAO to present my work at the 233rd meeting of the American Astronomical Society. This experience solidified my ability in coding as my most vital asset in performing astrophysical research.

After graduation, I knew I wanted to do post-baccalaureate research with my former professor, Dr. Desika Narayanan. Since he was already familiar with my expertise in programming, he hired me to perform significant upgrades to his code, Powderday. Powderday performs dust radiative transfer using snapshots of galaxy formation simulations. My work began by simply running the code, but only a week later I was developing upgrades. I met with Dr. Narayanan daily to facilitate this rapid progress. \textbf{Not only did I develop a new feature to convolve simulations with instrument filters, but I also led the development of a repository-wide dependency upgrade, enabling modern volumetric data analysis in yt 4.0.} This required working in parallel to all the other developers, ensuring that each change I made would be both backwards-compatible and future-thinking. In addition, I made significant contributions to the documentation of the code base, which was critical before public release. Our work led to a recent publication (Narayanan et al. 2020) for which I am listed as the 5th pre-alphabetical author. This project was my first introduction to large-scale development between many authors, and gave me a strong basis in numerical radiative transfer simulations, which I have made a central focus of my graduate career.

My work in radiative transfer solidified my interest in computational astrophysics as the basis for my future research. After being admitted to the University of Virginia, I began researching outflows from the surfaces of neutron stars with Dr. Shane Davis. I used the radiation magnetohydrodynamics (Rad-MHD) code Athena++ (Stone et al. 2019) to produce time-dependent models of atmospheric shocks caused by nuclear detonations in helium-burning layers of neutron stars. This approach surpassed the limitations of previous work: namely, steady-state models (Paczynski 1983) and those which utilized the diffusion approximation for radiation transfer in the outer layers of the burst (Yu et al. 2018). My method allows for a complex treatment of the problem, uniquely poised to evaluate assumptions commonly made in observational measurements of Type I X-ray bursts. Since these bursts are the predominant method for measuring the mass-radius relationship of neutron stars (Ozel et al. 2006), my work contributes directly toward justifying observational constraints on the neutron star equation of state---a central problem in the study of condensed matter and general relativity. My interest in this topic led me to take two elective courses in General Relativity at a graduate level, in which I have studied how relativistic effects can be taken into account in numerical models of these phenomena.

The summer after my first year of graduate school, I sought out a graduate research assistantship with joint mentorship between Dr. Shane Davis and Dr. Phil Arras, an expert in theoretical methods. I wanted to broaden my scope through this work, since I came to realize that a strong theoretical background is a prerequisite for meaningful computational work. Our work together on analytical and numerical solutions to Lyman-Alpha radiation transfer has formed the basis of my Research Statement, and has led to an upcoming publication (McClellan et al., \textit{in prep}). The work I have accomplished at UVa since May has ensured that the foundations of this project are well-established. I am on track to transition smoothly toward my Master's defense as I prepare to publish my findings. Past that point, I will develop a high-performance radiation hydrodynamics code using my Monte Carlo acceleration scheme. Though this work will undoubtedly form the basis of my PhD thesis, its effect on my career will be much more far-reaching. These experiences have narrowed my focus and honed my skill in radiative transfer astrophysics, equipping me with a set of computational tools I will continue to make use of for the duration of my career as an astrophysicist. As I look for post-doctoral research and staff scientist positions past graduation, my rich background in computational methods will guide the aims of my future work.

\thispagestyle{empty}
\noindent \textbf{\underline{Broader Impacts}} 


When the graduate student leading my first-ever programming workshop said, "Everyone starts this way. In a few years, you'll be an expert," I admitted I didn't believe him. But sure enough, he and I ended up working side-by-side as software developers on a major astrophysical source code. The early experiences in my research career were made possible by several graduate student mentors, who took time out of their busy days to teach me fundamental concepts in astronomy and basic research methods. A priority of mine is to pay this forward by helping younger students in the same way. \textbf{This is why I have taken on leadership of the Central Virginia Governor's School mentorship program, which introduces high school students to current topics in Astronomy with a single semester project. I will continue to lead this program until the completion of my PhD}, evaluating the experiences of participating students with exit surveys to be offered at the end of the program. My aim over the next 4 years is to diversify science by stimulating interest among marginalized youth via 1-on-1 mentorship, and to prepare them for possible careers in STEM through their research projects. In meetings I've had with the Astronomy admissions committee, the issue of diversity in the recruitment pool has been presented as a significant barrier to increasing representation in science. Therefore, through the CVGS mentorship program, I will facilitate the accessibility of STEM to under-represented minority students in order to boost inclusivity at this level of recruitment.

In my time on the Diversity, Equity, and Inclusion Committee (DEI), I have learned how invisible barriers restrict under-represented minority (URM) students' access to STEM and higher education as a whole. Though I always knew such barriers existed, I was disheartened to learn about specific instances my own peers had experienced, including gender-based discrimination, minority tokenization, and micro-aggressions. While the DEI has started exploring solutions on a departmental scale, I have made my voice heard on key issues to address the barriers facing current and future students, which I will outline below.

As a member of the LGBTQ+ community, when I began my work on equity and inclusion I realized I didn't know any other queer graduate students in my program. So, I created a private communication channel to connect us together, providing a safe space and support system for those who may not have one at home. Since the creation of the channel, we have bonded together as a sub-community that can share common experiences and help uplift each other through difficult circumstances. I will continue to extend invitations to each group of incoming students, acting as an outspoken proponent of protected spaces. As a representative on the DEI committee, \textbf{I will coordinate a series of 3 training sessions on pronouns, gender identity, and gender-based discrimination in STEM, provided by the UVa Women's Center and LGBTQ Center.}

In terms of addressing racial inequity directly, I was an early advocate for removal of the Physics and General GRE tests from our department's graduate astrophysics program application requirements. Among many other issues, such examinations represent a disproportionately large barrier for financially-disadvantaged students, and have since been removed from the graduate application. In addition, as Head TA of the Astronomy department, I negotiated a reduced workload for an international student who cannot receive pay from the University, and must work full-time to support himself during the pandemic. Though the actions taken on these issues represent a promising start to addressing inequity present in our department, \textbf{I am currently advocating to rework the structure of the admissions committee to address implicit bias, and to hire a diversity consultant to evaluate the racial climate of the department.} The climate assessment will include a survey period, in which participants will evaluate the successes and shortcomings of the departmental changes made to date.

These ongoing developments are a high priority commitment, and though I am proud of the progress that has been made so far, I see substantial work that remains to be done. It is important to realize that breaking down barriers isn't exclusive to future students who have yet to walk the path---through this work, I will help our current students facing difficulties \textit{now}.

\thispagestyle{empty}
\vspace{0.45cm}
\begin{adjustbox}{}
\begin{tabular}{ll}
    \textbf{References} \\
    Ransom 2001, Harvard University PhD Thesis &
    McClellan et al. 2019, AAS Meeting Abstracts vol. 233\\
    Narayanan et al. 2020, arXiv:2006.10757 &
    Stone et al. 2019, ascl:1912.005\\
    Paczynski 1983, ApJ, 267, 315 &
    Yu et al. 2018, ApJ, 863, 53\\
    Ozel et al. 2006, Nature, 441, 1115&
    McClellan et al. 2020, ApJ, \textit{in prep}
\end{tabular}
\end{adjustbox}


%\renewcommand\refname{\vskip -1cm}
%\bibliographystyle{ieeetr}
%\bibliography{ref}

\end{document}
